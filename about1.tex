\section*{Travels of Baron Munchausen}
\begin{quote}
\textit{How should I disengage myself? I was not much pleased with my awkward
situation -- with a wolf face to face; our ogling was not of the most pleasant
kind. If I withdrew my arm, then the animal would fly the more furiously upon
me; that I saw in his flaming eyes. In short, I laid hold of his tail, turned
him inside out like a glove, and flung him to the ground, where I left him.}
\end{quote}

On January 3 2011, the first working day of the year, OSP gathered around
Scribus. We wanted to explore framerendering, bootstrap Scribus and play
around with the incredible adventures of Baron von Munchausen.

The idea was to produce some kind of experimental result, a follow-up of to
our earlier attempts to turn a frog into a prince.

One day we asked Scribus team-members what their favourite Scribus feature
was. After some hesitation they pointed us to the magical framerender.

\section*{Framerenderer}
The Framerender is an image frame with a wrapper, a GUI, and a configuration
scheme. External programs are invoked from inside of Scribus, and their output
is placed into the frame!

By default, the current Scribus is configured to host the following friends:
LaTeX, Lilypond, gnuplot, dot/GraphViz and POV-Ray

Today we are working with Lilypond. We are also creating custom tools that
generate PostScript/PDF via Imagemagick and other command line tools.

\section*{Lilypond}
LilyPond is a music engraving program, devoted to producing the
highest-quality sheet music possible.  It brings the aesthetics of
traditionally engraved music to computer printouts.
