\documentclass{article}
\begin{document}
\section*{Text}
We typeset the first 5 chapters of the stories that Rudolph Erich Raspe wrote
in the voice of the legendary Baron Munchausen. In a rudimentary English, he
relates about incredible adventures that happen to him in rapid succession. We
started reading them because the passage where Munchausen brings himself out
of the swamp by pulling his own boots is often brought up when talking about
the Bootstrapping of a computer. Or is it him pulling himself out by his own
hair? We have yet to find this exact reference in the Project Gutenberg text.
Maybe someone can help? :-)

\section*{Typeface}
The shapes of Univers Else [3] are obtained from scanning printed textpages
that were optically composed by cheap phototypesetting machines 30 years ago.
With it's round angles, floating baselines and erratic kerning it has the kind
of delirious brashness necessary for typesetting stories of half horses,
wolves turned inside out like a glove, fifty ducks destroyed by one shot and
fighting lions with crocodiles.

\begin{quote}
\textit{I recollected that Turkey-beans grow very quick, and run up to an
astonishing height. I planted one immediately; it grew, and actually fastened
itself to one of the moon's horns. I had no more to do now but to climb up by
it into the moon, where I safely arrived, and had a troublesome piece of
business before I could find my silver hatchet, in a place where everything
has the brightness of silver\dots}
\end{quote}

\end{document}
